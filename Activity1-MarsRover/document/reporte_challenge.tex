% This is samplepaper.tex, a sample chapter demonstrating the
% LLNCS macro package for Springer Computer Science proceedings;
% Version 2.20 of 2017/10/04
%
\documentclass[runningheads]{llncs}
%
\usepackage{graphicx}
% Used for displaying a sample figure. If possible, figure files should
% be included in EPS format.
%
% If you use the hyperref package, please uncomment the following line
% to display URLs in blue roman font according to Springer's eBook style:
% \renewcommand\UrlFont{\color{blue}\rmfamily}

\begin{document}
%
\title{Challenge 1: Mars Rover}
%
%\titlerunning{Abbreviated paper title}
% If the paper title is too long for the running head, you can set
% an abbreviated paper title heres
%
\author{Arturo Burela \and
Horacio Rojas \and
Fernando Alcántara}
%
\authorrunning{F. Author et al.}
% First names are abbreviated in the running head.
% If there are more than two authors, 'et al.' is used.
%
\institute{Instituto Tecnológico y de Estudios Superiores de Monterrey}
%
\maketitle              % typeset the header of the contribution
%
\begin{abstract}
The present paper will explore how reactive agents work, the main characteristics of reactive agents and a comparison between individual and collaborative behaivor,
all of the above through a couple of simulations of the Mars explorer problem (Weiss, 1999).

\keywords{Multi Agent System  \and Distributed Artificial Intelligence \and Reactive Agents.}
\end{abstract}
%
%
%
\section{First Section}
\subsection{Introduction}

Reactive agents are a type of agent 


PUES AQUI VA LA INTRO BIEN PERRONA, HASTA ABAJO ESTÁN LAS PARTES QUE PIDIÓ EL PROFE, ESTE DOC ES UN TEMPLATE Y 
DEJO LA TABLA Y TODO LO DEMÁS NOMAS POR SI LAS NECESITAMOS EN ALGUN MOMENTO

iNTRO BIEN PERRONA


Please note that the first paragraph of a section or subsection is
not indented. The first paragraph that follows a table, figure,
equation etc. does not need an indent, either.

Subsequent paragraphs, however, are indented.


\section{Second Section}
\subsection{Aparatus}


\section{Third Section}
\subsection{Behavious}
\subsection{Experiments}


\section{Fourth Section}
\subsection{Conclusions}

\section{Manual}
asdasdasd

%
% ---- Bibliography ----
%
% BibTeX users should specify bibliography style 'splncs04'.
% References will then be sorted and formatted in the correct style.
%
% \bibliographystyle{splncs04}
% \bibliography{mybibliography}
%
\begin{thebibliography}{8}
\bibitem{ref_article1}
Author, F.: Article title. Journal \textbf{2}(5), 99--110 (2016)

\bibitem{ref_lncs1}
Author, F., Author, S.: Title of a proceedings paper. In: Editor,
F., Editor, S. (eds.) CONFERENCE 2016, LNCS, vol. 9999, pp. 1--13.
Springer, Heidelberg (2016). \doi{10.10007/1234567890}

\bibitem{ref_book1}
Author, F., Author, S., Author, T.: Book title. 2nd edn. Publisher,
Location (1999)

\bibitem{ref_proc1}
Author, A.-B.: Contribution title. In: 9th International Proceedings
on Proceedings, pp. 1--2. Publisher, Location (2010)

\bibitem{ref_url1}
LNCS Homepage, \url{http://www.springer.com/lncs}. Last accessed 4
Oct 2017
\end{thebibliography}
\end{document}
